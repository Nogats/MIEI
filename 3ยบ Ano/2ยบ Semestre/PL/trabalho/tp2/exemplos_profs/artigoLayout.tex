%
% Layout retirado de http://www.di.uminho.pt/~prh/ii02pd.html
%
%\documentclass[twocolumns]{article}
\documentclass{article}

\usepackage[portuges]{babel}
\usepackage[utf8]{inputenc}

\usepackage{graphicx}

\parindent=0pt
\parskip=2pt


\title{Um Exemplo de Artigo em \LaTeX}
\author{Sandra Lopes\thanks{Bolseiro da FCT}\\Escola FFF\\ Fafe
        \and
        Pedro Henriques\\Escola de Engenharia\\ Braga
        \and João Ferreira\thanks{Doutorando PDInf}\\ Escola de Direito }
\date{ (\today)}



\begin{document}

\maketitle

\begin{abstract}
\noindent Isto é um resumo do artigo.\\
O objectivo é sintetisar em 2 ou 3 parágrafos a ideia principal descrita no artigo.
\end{abstract}

\tableofcontents

\section{Introdução} \label{intro}

O artigo está organizado da seguinte forma.
Na Secção~\ref{backg} falo dos conceitos básicos necessários para perceber o texto.

Na Secção~\ref{llll} continuo a explicação.


aqui vai um exemplo de matemática $\alpha\   \oint   \Psi \Delta \pi \  \varphi\ \phi\ \wedge \biguplus\ {\AE}  $ inline

aqui vai um exemplo de matemática \[ \alpha\   \pi \  \varphi\ \phi\ \wedge \$ \] em destaque


\section{Background} \label{backg}
\section{Conceitos básicos em XXX}
\subsubsection{Conceitos básicos em XXX.1}
Aqui vai um exemplo de uma lista numerada
\begin{itemize}
\item primeira característica desta fase
\item segunda característica desta fase
\item terceira característica desta fase
\end{itemize}

\subsubsection{Conceitos básicos em XXX.2}
Seguem-se algumas definições fundamentais para se perceberem as ideias
defendidas a seguir:
\begin{description}
\item[conceito 1] respectiva definição 1
\item[conceito 2] descrição do  conceito 2
\end{description}

\subsection{Conceitos básicos em YYY}
Veja a Figura \ref{figuraA}.

\section{A Proposta} \label{llll}
Como se vê na Figura~\ref{figuraA} bla bla bla
    \begin{figure}
        \begin{center}
OOOOOOOO\\
BBBBBB\\
XXXX
            \caption{Legenda da Figura} \label{figuraA}
        \end{center}
    \end{figure}

\section{A sua Implementação}
A Figura \ref{figuraB}, que surge na pagina~\pageref{figureB}, mostra o esquema seguido     na
implementação.

Os      resultados finais desta implementação serão discutidos na Secção \ref{concl}.

\begin{figure}[htbp]
\begin{center}
            \includegraphics[width=\textwidth]{fotografia.jpg}
    \caption{Legenda da Imagem} \label{figuraB}
\end{center}
\end{figure}

\section{Conclusão} \label{concl}
Síntese do que foi dito.\\
Lista dos resultados atingidos:
\begin{itemize}
\item resultado 1
\item resultado 2
\end{itemize}
Conclusão final e Trabalho Futuro.

\end{document}
